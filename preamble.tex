\usepackage{iftex}
%% Разрешить компиляцию только с движком LuaTex
\makeatletter
\ifLuaTeX
\else
    \newlinechar 64\relax
    \errorcontextlines -1\relax
    \immediate\write20{@
        ************************************************@
        * LuaLaTex is required to compile this document.@
        * Sorry!@
        ************************************************}%
    \batchmode\read -1 to \@tempa
\fi
\makeatother

%% Для русификации достаточно подключить пакет fontspec и
%% выбрать Unicode шрифт в котором есть кириллические глифы. Ниже
%% основным шрифтом выбирается Unicode версия шрифта Computer Modern с заcечками
\usepackage{fontspec}
\setmainfont{CMU Serif}
\setsansfont{CMU Sans Serif}
\setmonofont{CMU Typewriter Text}

%% В LuaLaTeX альтернативой известного пакета babel является пакет polyglossia.
%% Теперь у нас будут переносы слов
\usepackage{polyglossia}
\setdefaultlanguage{russian}
\setotherlanguage{english}

\usepackage[autostyle]{csquotes} % Правильные кавычки в зависимости от языка

% ToDo:
% [] MathNote   : \tikzmatrix 
% [] MathNote   : smashable \sum* commands
% [] Layout     : Define colors
% [] Fix hidding enviroment

%Отключить предупреждения об кастномной использовании пакетов "You have requested package..."
\usepackage{silence}
\WarningFilter{latex}{You have requested package}

\usepackage{xparse}
%\usepackage{configuration/floats}
%\usepackage{configuration/layout}
%\usepackage{configuration/references}
%\usepackage{configuration/mathematics}

\sloppy
% Окружения для набора задач
%\input{configuration/teaching/preamble-common}
