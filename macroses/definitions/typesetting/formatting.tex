%%% Списки
\SetEnumitemKey{compact}{topsep=0pt, parsep=0pt, noitemsep}


%%%%% Подчеркивания и прочее

\ExplSyntaxOn
%\uline{}  --- просто подчеркивает текст, синоним \underLine
%\uline*{} --- проводит последнюю линию до конца, но нужно чтобы стоял \par в конце
\NewDocumentCommand{\uline}{s +m}{
    \IfBooleanTF{#1}{
        \AddToHook{para/end}[UnderlineTillEOL]{\hfill\phantom{.}}
        \underLine{#2 \par}
        \RemoveFromHook{para/end}[UnderlineTillEOL]
    }{\underLine{#2}}
}
%\iuline{}  --- тоже самое что и \uline, но абзацный отступ тоже подчеркивается
%\iuline*{} --- *-версия \iuline, с соответствующим поведением
\newlength\OldParIndentDim
\NewDocumentCommand{\iuline}{s +m}{%
    \setlength{\OldParIndentDim}{\parindent}
    \setlength{\parindent}{0pt}

    \AddToHook{para/begin}[UnderlineIdentationHook]{\hspace{\OldParIndentDim}}
    \IfBooleanTF{#1}{\uline*{#2}}{\uline{#2}}
    \RemoveFromHook{para/begin}[UnderlineIdentationHook]

    \setlength{\parindent}{\OldParIndentDim}
}

\box_new:N\l_lines_tmp_box
\cs_new:Npn\count_lines_number:nn#1#2{
    \vbox_set:Nn\l_lines_tmp_box{#1}
    \exp_args:No\int_set_eq:NN{#2}{
        \fp_to_int:n{\dim_ratio:nn{\box_ht:N\l_lines_tmp_box}{\baselineskip} + 0.5}
    }
}
\int_new:N\l_lines_count_int
\int_new:N\l_lines_target_int

\NewDocumentCommand{\ulines}{m+m}{%
    \int_set:Nn\l_lines_target_int{#1}
    \count_lines_number:nn{#2}{\l_lines_count_int}
    \iuline*{#2} \par
    \prg_replicate:nn{\int_max:nn{\l_lines_target_int - \l_lines_count_int}{0}}{\noindent \uline{\hfill\phantom{.}} \par}
}

\ExplSyntaxOff


\ExplSyntaxOn
%\underscoredline[linesize]{main text}[pos]{underscored text}
\NewDocumentCommand{\underscoredline}{so+mO{c}m}{
    % Обятально использовать параметр [t]! Чтобы не съезжало посреди текста
    \IfNoValueTF{#2}{
        \measureremainder{\whatsleft}
        \measureremainder{\whatsleft}\begin{minipage}[t]{\whatsleft}
            #3 \phantom{р}\par %\phantom{р} исользуется чтобы добавить глубину, иначе линия будет наезжать на текст, а так это выглядит хорошо.
            \noindent\hrule\smallskip
            \exp_args:Nx\tl_if_eq:nnTF{#4}{l}{\makebox[\linewidth]{\IfBooleanTF{#1}{}{\scriptsize} #5}}{}
            \exp_args:Nx\tl_if_eq:nnTF{#4}{r}{\hfill\makebox{\IfBooleanTF{#1}{}{\scriptsize}  #5}}{}
            \exp_args:Nx\tl_if_eq:nnTF{#4}{c}{\centering\makebox[\linewidth]{\IfBooleanTF{#1}{}{\scriptsize} #5}}{}
        \end{minipage}
    }{
        \begin{minipage}[t]{#2}
            #3 \phantom{р}\par %\phantom{р} исользуется чтобы добавить глубину, иначе линия будет наезжать на текст, а так это выглядит хорошо.
            \noindent\hrule\smallskip
            \exp_args:Nx\tl_if_eq:nnTF{#4}{l}{\makebox[\linewidth]{\IfBooleanTF{#1}{}{\scriptsize} #5}}{}
            \exp_args:Nx\tl_if_eq:nnTF{#4}{r}{\hfill\makebox{\IfBooleanTF{#1}{}{\scriptsize}  #5}}{}
            \exp_args:Nx\tl_if_eq:nnTF{#4}{c}{\centering\makebox[\linewidth]{\IfBooleanTF{#1}{}{\scriptsize} #5}}{}
        \end{minipage}
    }
}
\ExplSyntaxOff
